%
% Slide template for the FreeBSD Developer Summit.
% Take it or leave it :-)
%
% It requires LaTeX and LaTeX Beamer [1] to compile.
% pdfLaTeX is recommended for compilation as it produces
% PDF file immediately.
%
% $ pdflatex some.latex
%
%
% It is also recommended to convert images to PDF
% by using ImageMagick ("convert") before including them.
%
% [1] http://latex-beamer.sourceforge.net
%

\documentclass{beamer}

\usepackage{url}
\usepackage[english]{babel}
\usepackage{verbatim}
\usepackage{graphicx}
\usepackage{listings}

\mode<presentation>
{
  \definecolor{beamer@gker}{rgb}{0.8,0.0,0.0}
  \setbeamercolor*{structure}{fg=beamer@gker}
  \logo{\includegraphics[scale=0.5]{logo.pdf}}
}

\setbeamertemplate{footline}[text line]{%
  \parbox{\linewidth}{\vspace*{-8pt}
  some text
  \hfill
  \insertshortauthor
  \hfill
  \insertframenumber\ of \inserttotalframenumber}}
\setbeamertemplate{navigation symbols}{}

\title{FreeBSD package management system}
\author{Vsevolod Stakhov \\ \url{vsevolod@FreeBSD.org}}
\institute{\includegraphics[scale=0.5]{logo.pdf}}
\date{ruBSD conference \
December 14, 2013}

\begin{document}

\begin{frame}[plain]
  \titlepage
\end{frame}

\begin{frame}
\frametitle{Ports and packages}
Ports is the comprehensive system of source packages.
\begin{itemize}
  \item Mature.
  \item Clear and well defined.
  \item Simple (sometimes not).
\end{itemize}
\end{frame}

\begin{frame}[fragile]
\frametitle{Ports before pkg}
\begin{figure}[h!]
  \centering
  \includegraphics[width=0.95\textwidth]{q1.eps}
\end{figure}
\end{frame}

\begin{frame}
\frametitle{Planned ports and pkg interaction}
\begin{figure}[h!]
  \centering
  \includegraphics[width=0.99\textwidth]{q2.eps}
\end{figure}
\end{frame}

\begin{frame}
\frametitle{Ports and packages}
\begin{itemize}
  \item Ports are used to build packages.
  \item Dependencies are resolved by pkg, not make.
  \item Stable branch of ports has an appropriate stable branch of packages.
  \item Encourage users to install software from binary packages.
  \item But do not prevent them from building custom packages from the ports.
\end{itemize} 
\end{frame}

\begin{frame}
\frametitle{Repositories creation}
\begin{figure}[h!]
  \centering
  \includegraphics[width=0.95\textwidth]{q3.eps}
\end{figure}
\end{frame}

\begin{frame}
\frametitle{Pkg architecture}
\begin{figure}[h!]
  \centering
  \includegraphics[width=0.95\textwidth]{q4.eps}
\end{figure}
\end{frame}

\begin{frame}
\frametitle{The current problems with pkg}
\begin{itemize}
  \item Legacy ports support (with no staging, for example).
  \item Plain dependencies style.
  \item Immature codebase.
  \item Naive solver.
\end{itemize}
\end{frame}

\begin{frame}
\frametitle{The problems of the solver in pkg}

\begin{itemize}
\item Absence of conflicts resolving/handling;
\item No alternatives support;
\item Can perform merely a single task: install, upgrade, remove or autoremove,
so install task cannot remove packages for examle.
\end{itemize}

\end{frame}


\begin{frame}
\frametitle{The existing work}

There are many examples of solvers, for example:
\begin{itemize}
  \item \texttt{libsolv} - the complete solver and package management library;
  \item \texttt{Apt} solvers interface;
  \item \texttt{Mancoosi} - a European research project that compares and study
  different solvers;
\end{itemize}

\end{frame}

\begin{frame}[fragile]
\frametitle{External solvers}
To interact with an external solver we have chosen CUDF format used in the
Mancoosi project:
\bigskip
{\small
	\begin{verbatim}
	package: devel/libblah
	version: 1
	depends: x11/libfoo

	package: security/blah
	version: 2
	depends: devel/libblah
	conflicts: security/blah-devel
	
	\end{verbatim}
}
\end{frame}

\begin{frame}
\frametitle{We need an internal solver!}

Alternatives:
\begin{itemize}
  \item Write own logic of dependencies and conflicts resolution?
  \pause
  \item Use some existing solution?
  \pause
  \item Use some known algorithm?
  \pause
\end{itemize}
\bigskip
{\large Use SAT solver for packages management}
\bigskip
\[(x_1 \| \neg x_2 \| x_3) \& (x_3 \| \neg x_1) \& (x_2)\]
\end{frame}

\begin{frame}
\frametitle{Making a SAT problem}
\begin{itemize}
  \item Assign a variable to each package: 
  package A $\to a_1$, package B $\to a_2$
  \item Interpret a request as a set of unary clauses:
  \begin{itemize}
    \item Install/Upgrade package A $\to (a_1)$
    \item Delete package B $\to (\neg a_2)$
  \end{itemize}
  \item Convert dependencies and conflicts to disjuncted clauses
\end{itemize}

\end{frame}

\begin{frame}
\frametitle{Converting dependencies and conflicts}
\begin{itemize}
  \item If package A depends on package B (versions $B_1$ and $B_2$), then we
  can either have package A not installed or any of B installed:
  \bigskip
\[(\neg A \| B_1 \| B_2)\]
  \item If we have a conflict between versions of B ($B_1$, $B_2$ and $B_3$)
 then we ensure that merely one version is installed:
  \bigskip
\[(\neg B_1 \| \neg B_2) \& (\neg B_1 \| \neg B_3) \& (\neg B_2 \| \neg
B_3)\]
\end{itemize}
\end{frame}

\begin{frame}
\frametitle{The solving of SAT problem}

Some rules to follow to speed up SAT problem solving.
\begin{itemize}
  \item Trivial propagation - solve unary clauses;
  \item Unit propagation - solve clauses with only a single unsolved variable;
  \item Conflicts learning - if we assign some free variable and detect a
  conflict during unit propagation, we can fallback and learn that this variable
  must be negated;
  \item Package specific assumptions. 
\end{itemize}
\end{frame}

\begin{frame}
\frametitle{SAT problem propagation}
\begin{itemize}
  \item Trivial propagation - direct install or delete rules.
  \bigskip
  \[(\neg A \| B_1 \| B_2) \& (A) \&(\neg C) \& (\neg A \| \neg D)\]
  \bigskip
  Assume A as \textit{true} and C as \textit{false}.
  \item Unit propagation - simple depends and conflicts.
  \bigskip
  \[(\neg A \| B) \& (A) \&(\neg C) \& (\neg A \| \neg D)\]
  \
  Assume B as \textit{true} and D as \textit{false}, as A is true and it
  depends on B and conflicts with D.
\end{itemize}
\end{frame}

\begin{frame}
\frametitle{Conflicts driven learning}
To handle alternatives it is required to test all variables unassigned:
\begin{enumerate}
  \item full depth-first enumeration of possible values;
  \item fallback if a conflict found;
  \item remember which assignment caused conflict;
  \item make negative assignment for the learned variable and go to the first
  step.
\end{enumerate}
\end{frame}

\begin{frame}
\frametitle{Package specific assumptions}
Pure SAT solvers cannot deal with package management as they do not consider
several packages peculiarities:
\begin{itemize}
  \item try to keep installed packages (if no direct conflicts);
  \item do not install packages if they are not needed;
  \item prefer high priority packages and repositories over low priority ones.
\end{itemize}
These options also improve SAT performance providing a good initial assignment.
\end{frame}

\begin{frame}
\frametitle{Packages universe}
We convert all packages involved to a packages universe of the following
structure:
\begin{figure}[h!]
  \centering
  \includegraphics[height=0.7\textheight]{q5.eps}
\end{figure}
\end{frame}

\begin{frame}
\frametitle{Package management task}
\begin{itemize}
  \item A request is splitted to install/upgrade and delete requests which
  could be passed simultaneously to the solver;
  \item A conflicts between packages are detected with a repository creation;
  \item All depends, reverse and conflicts of the requested packages are
  analyzed and the package universe is created;
  \item Each package is defined by its name and the digest of significant
  fields (version, options and so on);
\end{itemize}
\end{frame}

\begin{frame}
\frametitle{Solvers and Pkg}
\begin{itemize}
  \item Pkg may pass the formed universe to an external CUDF solver:
  \begin{itemize}
    \item convert versions;
    \item format request;
    \item parse output.
  \end{itemize}
  \item Alternatively the internal SAT solver may be used:
  \begin{itemize}
    \item convert the universe to SAT problem;
    \item formulate request;
    \item ???
    \item PROFIT
  \end{itemize} 
\end{itemize}
\end{frame}

\begin{frame}
\frametitle{Perspectives}

\begin{itemize}
  \item Using pkg solver for ports management.
  \item New dependencies and conflicts format.
  \item Better support of multiple repositories .
  \item Provides and alternatives.
  \item Test different solvers algorithms using CUDF.
\end{itemize}
\end{frame}

\begin{frame}
\frametitle{New dependencies format}
%% Picture
\end{frame}

\begin{frame}
\frametitle{Alternatives}
%% Picture
\end{frame}

\begin{frame}
\begin{center}
\includegraphics{logo.pdf}
{\Large Thank you for your attention!} \\
\emph{Questions?}
\end{center}
\end{frame}

\end{document}
